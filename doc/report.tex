\documentclass[12pt,a4paper]{article}
\usepackage[british]{babel}
\usepackage[utf8]{inputenc}
\usepackage[T1]{fontenc}
\usepackage{booktabs}
\usepackage{hyperref}
\usepackage{tikz}
\usepackage{verbatim}

\title{Final Project: The Report}
\author{Ingibergur Sindri Stefnisson \and Gabríel Arthúr Pétursson \and Kristján Árni Gerhardsson
	\footnote{The order of the names listed were determined by Procrastination.}}

\begin{document}

\maketitle

\clearpage

%% an introduction, which introduces the topic of the project, motivates it (= 
%% explain why you believe this is interesting) and states the
%% purpose of the project (what you wanted to achieve),
%% a  description of what you did, which methods you used and why you chose these methods
%% a presentation, interpretation and discussion of the results (e.g., data from 
%% experiments you did and what the data means and how it compares %to other experiments or what you expected)
%% a conclusion (summary of the project's goals, results and their implications)
%% some suggestions for future work

\section*{Introduction}

The topic of our project was to make a Sudoku solver and generator. What interested
us in doing this is that all of us like Sudoku and we wanted to see how we would
go about making our own Sudoku program with a solver and a generator.\\

First a little backstory on Sudoku.\\

Sudoku was popularized in 1986 in Japan but it didn't become internationally known
until 2005. Since then it has been puzzling people all over the world. 
There are quite a lot of variations of the Sudoku puzzle in existence. But our 
solver/generator was made for the normal type of Sudoku.\\

What we originally wanted to achieve was just a simple solver for 9x9 Sudoku puzzles.
What we found out though, that 9x9 puzzles are very easy to bruteforce by using
a backtracking search algorithm. So we decided to go deeper and see if we could
make a solver that could solve 16x16 puzzles quickly, since the backtracking algorithm
is quite useless for those without modifications.\\

So with the simple backtracking search we started by changing it to have forward 
checking and later on constraint propagation.

\section*{Description}

\ldots

\section*{Results}%presentation, interpretation and discussion of the results.

\ldots

\section*{Conclusion}

\ldots

\section*{Future Suggestions}

\ldots

\end{document}
