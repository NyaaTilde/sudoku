\documentclass[12pt,a4paper]{article}
\usepackage[british]{babel}
\usepackage[utf8]{inputenc}
\usepackage[T1]{fontenc}
\usepackage{booktabs}
\usepackage{hyperref}
\usepackage{tikz}
\usepackage{verbatim}

\title{Final Project: The Report}
\author{Gabríel Arthúr Pétursson \and Ingibergur Sindri Stefnisson \and Kristján Árni Gerhardsson
	\footnote{The order of the names listed were determined by their amount of procrastination.}}

\begin{document}

\maketitle

\clearpage

%% an introduction, which introduces the topic of the project, motivates it (= 
%% explain why you believe this is interesting) and states the
%% purpose of the project (what you wanted to achieve),
%% a  description of what you did, which methods you used and why you chose these methods
%% a presentation, interpretation and discussion of the results (e.g., data from 
%% experiments you did and what the data means and how it compares %to other experiments or what you expected)
%% a conclusion (summary of the project's goals, results and their implications)
%% some suggestions for future work

\section*{Introduction}

The topic of our project was to make a Sudoku solver and generator. What interested
us in doing this is that all of us like Sudoku, and we wanted to see how we would
go about making our own Sudoku program with a solver and a generator.

First a little backstory on Sudoku.

A standard Sudoku puzzle consist of a nine by nine grid, split into nine three
by three boxes. Each square has to be filled with a number between one and nine.
Some of these squares have already been filled into to ensure that there exists
only one unique solution.

There are three rules: each row, column, and box must contain all the numbers
one through nine. Though these rules seem simple, there exists
6,670,903,752,021,072,936,960 valid Sudoku puzzles.

Sudoku was popularized in 1986 in Japan but it didn't become internationally known
until 2005. Since then it has been puzzling people all over the world.
There are quite a lot of variations of the Sudoku puzzle in existence. But our 
solver and generator was made for the normal type of Sudoku.

What we originally wanted to achieve was just a simple solver for 9x9 Sudoku puzzles.
What we found out though, that 9x9 puzzles are very easy to bruteforce by using
a backtracking search algorithm. Thus, we decided to go deeper and see if we could
make a solver that could solve 16x16 puzzles quickly, since the backtracking algorithm
is quite useless for those without modifications.

%%So with the simple backtracking search we started by changing it to have forward 
%%checking and later on constraint propagation.

\section*{Description}%% a  description of what you did, which methods you used and why you chose these methods

So first off, we are using the programming language Vala and utilizing the GTK+ toolkit for our graphical interface. %%It is simply the best dudududu
%%better than all the rest dududu%%talk about vala to make this longer

What we started off by doing is modeling Sudoku as a Constraint Satisfaction Problem,
where each cell is a variable, and the domains are the numbers possible to put in each
cell.

What is needed for a CSP is first and foremost a Backtracking search. Thus, we started
by implementing a simple version of a Sudoku Backtracking search, which worked by
itself quite well for 9x9 Sudoku puzzles. Solving even what is called the hardest
Sudoku puzzle ever in under 60~milliseconds. So from there we started thinking about bigger Sudoku puzzles.

We wanted to make something that could solve 16x16 puzzles.
Then we started implementing forward checking. And after forward cheking we 
implemented Constraint Propagation and made a fair heuristics function. With those changes we should be able to solve all valid 16x16 puzzles.

\subsection*{Programming Language}

For this project we made use of the Vala programming language. Vala is an
Object-Oriented programming language, similar to C\# and C++. The Vala code is
compiled into regular C code and bound together with the glib library for object
creation and management. Then the C code is compiled like regular C code, using
the GCC compiler package. We chose this language because of the C-like performance
it has, even though it is an OOP language. Because of this it suits quite well
for searching problems.

\subsection*{Code structure}

So the way our code is structured is we have a class \verb+Board+ that has arrays of \verb+CellList+s that store the rows, columns and boxes, and an array of all \verb+Cell+s in the \verb+Board+.

In there we have a function to solve the Sudoku.

In the \verb+CellList+ we have an array of cells for which cells are in that list's row, column, or box.

The \verb+Cell+ class has a bitmask used for forward checking,
what number belongs in the cell, along with integers represeting which row and
column the cell belongs to.

%% Add more here perhaps

\subsection*{Backtracking Search}

The Backtracking Search starts by checking if there is any unassigned cell. If it 
finds no unassigned cell it returns \verb+true+.

If it finds a unassigned cell it checks for each number in Sudoku if it is safe 
to put that one there. If it is safe to put it there it sets the cell's value to that 
number and recursively goes into the function again.

\subsection*{Forward Checking}

The Forward Checking algorithm adds on to the backtracking search by making each 
cell have a bitmask representing which values are possible for that cell. If 
the value in a particular location in the bitmask is zero, the cell cannot
contain that number. We maintain those bitmasks for each cell.

\subsection*{Constraint Propagation}

What the Constraint Propagation does is, to check for every cell if it has only
one possible solution.

Once a cell that must have one unique solution has been found we propagate the
exclusion of that possibility to all the cells in the same row, column, and box.
This can cause a cascading effect if another cell only has a single solution
left as a result of the exclusion of that number. This will continue to recurse
until all nodes with a unique solution have been filled.

This causes by far the biggest improvement in performance of all the optimizations,
in all the algorithms. In some cases it can solve a puzzle without having to
expand a single state, just by propagating the initial values through all the
cells lists.

\subsection*{Heuristic}

We are using Most constrained Variable Heuristic to choose which variable should
be assigned a value next. It minimizes the branching factor by quite alot.
What it does is pick the variable that has the fewest domains and uses it next.

After Most constrained Variable we try to find the least constrained value. As in 
the value in that variable that is on the fewest variables, since it's the one most
likely to have the solution.

\section*{Results}%presentation, interpretation and discussion of the results.
%experiments you did and what the data means and how it compares %to other experiments or what you expected)

Our first experiments was with just normal backtracking; we wanted to see how fast
the backtracking was at tackling 9x9 Sudoku puzzles. It came as a surprise that
it solved all 9x9 Sudoku puzzles in around 50~milliseconds.

When we tried to solve 16*16 puzzles though it could not solve it over a few days
running.

\subsection*{Generator}

The generation uses a rather trivial algorithm to generate puzzles. It starts
off by filling the first row with random numbers from 1 to $n$, for an $n$ sized
sudoku grid. Then it solves this puzzle. After solving the puzzle it will remove
a certain amount of numbers from the grid. For magnitude 3 sized boards it will
try to see if this new board has a unique solution, and generate a new one if
it is not so. Due to this being a trivial slow algorithm, this would take way
to long for anything greater than magnitude 3 boards, and so will just simply
return a board which might have more than 1 solution.

\subsection*{Experiments}

We did a few experiments with different difficulty of puzzles and different sizes.

With an easy 9x9 Sudoku the backtracking algorithm expanded around 4000 states.
The forward checking expanded around 2000 states. The constraint propagation 
algorithm allowed us to do most easy puzzles without expanding any states or at most
somewhere around 2.

With the medium 9x9 we tried the backtracking algorithm expanded
from 10-15 thousand states and the forward checking one around 3-5 thousand states
The constraint propagation algorithm with our heuristic was expanding somewhere around 120-150 states.

With the hard 9x9 Sudoku the backtracking algorithm expanded around 100 thousand
states. The forward checking expanded around 15000. The constraint propagation
algorithm expanded around 50-60 states.

The so called World's hardest sudoku by the Telegraph in the UK was the hardest 9x9
for our constraint propagation algorithm but it still managed to do it by expanding
only 1849 states. The forward checking algorithm did it in 22067 and the Backtracking one
did it in 49558 states.

But now comes the fun part the 16x16 puzzles. The backtracking search and the 
standard forward checking was completely unable to solve any of them that we found.

The easy 16x16 was solved in under a millisecond and with 0 states expanded.
The propagation did it all.

The hard 16x16 we used could be solved in around 380 thousand states.

The hardest 16x16 puzzle we could find, we were unable to solve in over 40 million 
states tried. :-(

\pagebreak

\section*{Conclusion}

Our conclusion to this project is that while Sudoku is a hard Constraint
Satisfaction problem it is not at all impervious to search problems. All 9x9
Sudoku puzzles we tried were handled quite easily, most in under 200 states
expanded. Our algorithm managed to solve nearly all 16x16 puzzles except for
the incredibly hard ones. And we can solve the easier 25x25 puzzles quite
quickly.

We made a sweet little GUI to use with it and a simple generator to make
puzzles.

\subsection*{Future suggestions}

Using constraint propagation with a good heuristic is a good way to solve small
and medium sized sudoku puzzles. However, generation requires a little more
specific algorithms, rather than just the trivial one we implemented. It would
be essential to include a more advanced generation algorithm for future versions
of the game.

\end{document}
